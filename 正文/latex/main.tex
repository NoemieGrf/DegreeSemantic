\documentclass{ctexart}

\usepackage{geometry}

\geometry{a4paper, right=3cm, left=3cm, top=2cm, bottom=2cm}

\usepackage{color}
\usepackage{soul}
\usepackage{amsmath}
\usepackage[british]{babel}
\usepackage[utf8]{inputenc}
\usepackage{epstopdf}
\usepackage{csquotes}
\usepackage[hidelinks]{hyperref}
\usepackage[
    style=apa,
    backend=biber,
    sortcites=true,
    sorting=nyt,
%    isbn=false,
%    url=false,
%    doi=false,
%    eprint=false,
    hyperref=false,
    backref=false,
%    firstinits=false,
]{biblatex}

% maps apacite commands to biblatex commands
\let \citeNP \cite
\let \citeA \textcite
\let \cite \parencite

\bibliography{reference}

\usepackage[T1]{fontenc}

% Required for inserting images
\usepackage{graphicx} 

\pagestyle{plain} % 页码放下面
\title{Semantics and Syntax of Degree Construction in and Mandarin}
\date{\vspace{-10ex}}

\usepackage{enumitem}
\setlist[enumerate,1]{label=(\arabic*)}
\setlist[enumerate,2]{label=(\alph*)}

%%%%%%%%%%%%%%%%%%%%%%%%%%%%%%%%%%%%%%%%%%%%%%%%%%%

\begin{document}

\thispagestyle{empty} % 这一页清空

\begin{center}

Name: $<<$* Zhang Yi Feng *$>>$ \\
Student ID: $<<$* 1155166448 *$>>$

\end{center}

{\let\newpage\relax\maketitle}

% 简介一下比较,给出中英例子,中文比较级分类
% 主要要说明清楚,positive comparative 这几个分类

\section{Introduction}

\setcounter{page}{1}

Gradability is an essential concept in the studies of adjectives, which classifies adjectives into two major classes, non-gradable adjectives, like \textit{British}, etc. and gradable adjectives, like \textit{tall}, \textit{long}, etc. Gradable adjectives are important role to build degree comparative constructions in a language. In this section, this paper will go through a batch of examples of comparatives and categorize them by their different structures, and this classification will lead semantic and syntactic analysis in next sections. 

\subsection{Simple degree structure and their classification}

A simplest example in English is shown in \ref{degree_construction_example}, in this sentence, word "\textit{than}" is a comparative marker whose job is introduce strict partial relation meaning "greater than" \hl{ref here}, and suffix "-\textit{er}" is another comparative marker whose function is still controversial. Some researchers believe that "-\textit{er}" is just a word in specifier position without crucial function \cite{rullmann1995,von1984a,heim1985,bhatt2004}, another kind of view treat this word as a head of Degree Phrase(DegP) \cite{corver1990,kennedy1997,grano2012}. Different form English, Mandarin is considered as a single mark language in comparative expression \cite{bobaljik2012,grano2012}. In most of cases, this mark is morpheme "\textit{bi}" such as example in \ref{degree_construction_example_Mandarin}, and there is no inflection of adjectives in Mandarin like "-\textit{er}" in English \cite{guo2012}. 

\begin{enumerate}
    \item \label{degree_construction_example}
    John is taller than Marry.
\end{enumerate}

\begin{enumerate}[resume]
    \item \label{degree_construction_example_Mandarin}
    John bi Marry gao.  \\
    \textit{John than Marry tall.} \\
    John is taller than Marry.
\end{enumerate}

Another important part of comparative structure is differential phrase(DP), whose job is giving the differential scale of context individuals. A comparative structure with differential phrase in English is shown in \ref{degree_construction_dp_example}, and Mandarin version is shown in \ref{degree_construction_dp_example_Mandarin}.

\begin{enumerate}
    \item \label{degree_construction_dp_example}
    John is 2 meters taller than Marry.
\end{enumerate}

\begin{enumerate}[resume]
    \item \label{degree_construction_dp_example_Mandarin}
    John bi Marry gao 2 mi.  \\
    \textit{John than Marry tall 2 meters.} \\
    John is 2 meters taller than Marry.
\end{enumerate}

The differential phrase in examples above are specific differential phrase, which give a accurately value of scale about gradable adjective. Another type of differential phrase gives vague value of scale. When this kind of differential phrase appears in comparative meaning, there is always a standard scale existed in context, and this vague value is either bigger than the standard value, or smaller than the standard value. Like example in \ref{dp_big_vague_example}, morpheme "\textit{hen duo}" is a big value vague differential phrase, which means, John is not only taller than Marry, but also, the difference of their height is lager than a standard value. This standard value is given in context, it is a consensus between speakers. Similarly, a small value vague differential phrase is shown in \ref{dp_small_vague_example}. In Mandarin, vague differential phrase can be complex, many researchers \cite{lin2014,li2015} do deeply investigation on it, \ref{dp_value_big_vague_example} and \ref{dp_value_small_vague_example} show such type of differential phrase which appears as "vague prefix + accurately value" or "accurately + vague suffix".

\begin{enumerate}[resume]
    \item
    \begin{enumerate}[ref=(\arabic{enumi}\alph*)]
        \item \label{dp_big_vague_example}
        John bi Marry gao hen duo.\\
        \textit{John than Marry tall much.} \\
        John is much taller than Marry.

        \item \label{dp_small_vague_example}
        John bi Marry gao yi dian.\\
        \textit{John than Marry tall a little.}\\
        John is a little taller than Marry.

    \end{enumerate}
\end{enumerate}

\begin{enumerate}[resume]
    \item
    \begin{enumerate}[ref=(\arabic{enumi}\alph*)]
        \item \label{dp_value_big_vague_example}
        John bi Marry gao liang mi duo.\\
        \textit{John than Marry tall 2 meters more.} \\
        John is more than 2 meters taller than Marry.

        \item \label{dp_value_small_vague_example}
        John bi Marry gao bu dao liang mi.\\
        \textit{John than Marry tall less 2 meters.} \\
        John is less than 2 meters taller than Marry.
    \end{enumerate}
\end{enumerate}

In comparative structure built by gradable adjectives, sometimes, the compare target is not a specific individual. This paper propose that there are three categories of compare target, which are "single individual comparison", "individual set comparison" and "specific value comparison". All examples mentioned above is single individual comparison. \ref{specific_value_comparison_example} gives a example of specific value comparison, in which compare target is not a individual. \ref{individual_set_comparison_example} shows a example of individual set comparison, and when it comes to compare with a individual set, the expression of sentence is changing from comparative meaning to superlative meaning.

\begin{enumerate}[resume]
    \item
    \begin{enumerate}[ref=(\arabic{enumi}\alph*)]
        \item \label{specific_value_comparison_example}
        John bi 2 mi gao.\\
        \textit{John than 2 meters tall.} \\
        John is taller than 2 meters.

        \item \label{individual_set_comparison_example}
        zai yi ban, John zui gao.\\
        \textit{in one class, John most tall.}\\
        John is tallest in class one.

    \end{enumerate}
\end{enumerate}

Besides comparative meaning and superlative meaning, the gradable adjectives can also build positive meaning and equative meaning. \ref{positive_meaning_example} shows a positive meaning example and \ref{equative_meaning_example} provides a equative meaning example. These two types is built by gradable adjective, but they do not have comparative meaning. In traditional research \hl{ref here}, the forms of comparative structure are usually divided into four categories, which are positive form, comparative form, equative form and superlative form. But form this paper's view, there are only tow forms, positive form and comparative form. The only condition of this kind of classification is, whether the gradable adjective in sentence provides a value comparison or a value assignment, namely:

\begin{itemize}
    \item[1.] A degree structure is comparative form if and only if this form's gradable adjective express "comparison" meaning.
    \item[2.] A degree structure is positive form if and only if this form's gradable adjective express "assignment" meaning.
\end{itemize}

Under this approach, \ref{positive_meaning_example} assigns a accurately value "\text{2 mi}" to another individual's height. When a scale-related group of words appears in positive form, this paper call it not differential phrase, but measure phrase(MP). Many researchers do not separate this two concepts very clear. Under out discussion, differential phrase appears in comparative form, measure phrase appears in positive form. 

\ref{equative_meaning_example} gives a expression that the value of John's height is assigned by the value of Marry's height. There is no comparative meaning so that the equative meaning is belong to positive form. Superlative meaning in \ref{individual_set_comparison_example} gives a comparison between the value of John's height and the values of others' height in class one. It is still a comparison, which makes superlative meaning classified into comparative form.

\begin{enumerate}[resume]
    \item
    \begin{enumerate}[ref=(\arabic{enumi}\alph*)]
        \item \label{positive_meaning_example}
        John gao 2 mi.\\
        \textit{John tall 2 meters.} \\
        John is 2 meters tall.

        \item \label{equative_meaning_example}
        John he Marry yi yang gao.\\
        \textit{John and Marry same tall.}\\
        John is as tall as Marry.

    \end{enumerate}
\end{enumerate}

By the perspective of gradable adjectives providing a "comparison" or "assignment", we can conclude some principals to depart superlative meaning from comparative form, and to depart equative meaning from positive form.

\begin{itemize}
    \item[1.] A comparative form expresses a superlative meaning if and only if this form is a individual set comparison.
    \item[2.] A positive form is a positive meaning if and only if this form has a explicit measure phrase.
\end{itemize}

Next phenomena should be noticed is the visibility of compare target. All examples above have explicit compare individual, but this target can be ignored when context has a default individual or a standard value. See example in \ref{diff_tradition_comparative_example}, in traditional research, this example is classified in positive form with positive meaning \hl{ref here}, but by this paper's approach, this example is comparative form because the gradable adjective bears the comparison function, rather than the assignment function. So the problem raises up which is that there is no compare target. Actually, there do exist a compare target under this context. The example expresses "\textit{very tall}", means that there is a standard height between speakers, and the value of John's height is greater than that standard. So this "specific value compare target" is actually a implicit compare target.

\begin{enumerate}[resume]
    \item \label{diff_tradition_comparative_example}
    John hen gao. \\
    \textit{John very tall.} \\
    John is very tall.
\end{enumerate}

Up to know, it is time to summarize all kinds classification mentioned above. First of all, a simple degree structure built by gradable adjectives can be classified into two categories, positive form and comparative form. If gradable adjectives bear value assignment function, sentence can be seen as positive form, and when gradable adjectives bear value comparison function, the degree structure can be seen as comparative form.In positive form, sentence can express positive meaning and equative meaning; and in comparative form, sentence can express comparative meaning and superlative meaning.

In comparative form, there are three different angles to classified degree structure. 

\begin{itemize}
    \item[1.] The differential phrase is explicit or implicit.
    \item[2.] The compare target is specific value or single individual or individual set.
    \item[3.] The compare target is explicit or implicit.
\end{itemize}

After there classifications defined, there are serval rules we can conclude in Mandarin.

\begin{itemize}
    \item[1.] Explicit differential phrase and set individual compare target do not appear at same time. Because explicit differential phrase is generated between two values, one from subject and the other from compare target, but a set of individual provides much more than one.
    \item[2.] If a comparative form has explicit differential phrase and compare target is specific value, this compare target can not be implicit. Because implicit compare target always provides context standard value, which is conflict with explicit specific value.
\end{itemize}

\subsection{Complex degree structure}

In last part of this section, this paper gives a fundamental view about simple gradable adjectives degree structures. But in Mandarin, a degree expression can be much complex. And this part, first, a important strategy to forming comparatives in Mandarin will be introduced which is called transitive comparative, and then, some of Mandarin crucial comparative morphemes and their properties will be discussed.

\subsubsection{Transitive degree structure}



\subsubsection{Complex comparative morpheme}



\section{Former Research}



\section{Semantic analysis in Mandarin}

In order to take a deep investigation about phenomena mentioned in last section, we shall go back to the really beginning: the lexical entry of gradable adjectives. 

\subsection{Lexical entry of gradable adjective}

For our knowledge, there are two typical hypotheses of what this lexical entry should look like, which are shown in \ref{tallLE}. The definition of lexical entry of gradable adjectives is literally important, because different definitions of adjectives' lexical entry always lead totally different results in semantics and syntax just like Butterfly Effect.

\begin{enumerate}[resume]
    \item \label{tallLE}
    
    \begin{enumerate}[ref=(\arabic{enumi}\alph*)]
        \item \label{tallLE_a} 
        $[\![tall]\!]=\lambda d \lambda x.[Height(x) \geq d]$
    
        \item \label{tallLE_b} 
        $[\![tall]\!]=Height(x)=d$
    
    \end{enumerate}
\end{enumerate}

\ref{tallLE_a} is a traditional lexical entry of gradable adjectives, and many researchers believe in it (\colorbox{yellow}{refhere}). Under this definition, gradable adjective \textit{tall} is considered as a relation of "greater equal", which is true when an individual $x$ as input and $x$'s height is at least as great as $d$. A simple application of this kind of lexical entry definition to \ref{old_school_def_example} is shown in \ref{old_school_def_example_LE}.

\begin{enumerate}[resume]

    \item \label{old_school_def_example} 2 meters tall.

\end{enumerate}

\begin{enumerate}[resume]

    \item \label{old_school_def_example_LE}
    
    \begin{enumerate}[ref=(\arabic{enumi}\alph*)]
        
        \item $[\![tall]\!]=\lambda d \lambda x.[Height(x) \geq d]$
        \item $[\![2 \enspace meters \enspace tall]\!]=\lambda x.[Height(x) \geq 2 \enspace meters]$

    \end{enumerate}

\end{enumerate}

Actually, lexical entry definition like \ref{tallLE_a} remains a problem with corresponding examples shown in \ref{old_school_def_problem}. In English, sometimes measure phrase can not directly combine with negative-pole adjectives. We can say someone is "\textit{2 meters tall}", but can not say someone is "\textit{2 meters short}". But situation changes when suffix \textit{er} shows up in \ref{old_school_def_problem_correct}, "\textit{2 meters shorter}" is a correct usage in English. Besides, this phenomenon is also a language-specific problem, in Japanese, we can not combine measure phrase even with \textit{segatakai}(\textit{tall}). Traditional lexical entry of gradable adjectives has no ability to explain this phenomenon.

\begin{enumerate}[resume]

    \item \label{old_school_def_problem}
    
    \begin{enumerate}[ref=(\arabic{enumi}\alph*)]
        
        \item 2 meters tall. 10 years old.
        \item *2 meters short. *10 years young.
        \item \label{old_school_def_problem_correct} 2 meters shorter. 10 years younger.

    \end{enumerate}

\end{enumerate}

Based on the problem mentioned above, there is another group of researchers propose that the gradable adjectives' lexical entry should not encode the partial ordering relation, instead, the gradable adjectives should reveal the original property of an individual, which means \textit{tall} should simply illustrate the height of an individual. Thus the lexical entry of gradable adjectives should looks like \ref{tallLE_b}, in which there is just a measure function. 

Under this assumption, the partial ordering relation still needs a place to be introduced, if not, there will be a lexical entry type mismatch. The type of measure function \textit{tall} is $<e,d>$, but measure phrase is type $d$, so gradable adjectives are no way to composite with measure phrase on account of type-theoretic. To resolve this problem, Svenonius\cite{svenonius2006} claims that there is a null operator whose semantic function is linking the lexical entry of gradable adjectives and the lexical entry of measure phrase, and syntactic function is to introduce a degree argument and bear the mission of introducing the "greater equal" meaning. The denotation of this null operator is spelled out in \ref{gao_nop_LE}, and the \ref{gao_nop_LE_c} shows the composition of null operator and gradable adjective, which is same with \ref{tallLE_a}. The lexical entry of \ref{gao_nop_LE_c} is $<d,<e,t>>$, which can combine with measure phrase with out any type conflict.

\begin{enumerate}[resume]
    \item \label{gao_nop_LE}

    \begin{enumerate}[ref=(\arabic{enumi}\alph*)]
        \item \label{gao_nop_LE_a} 
        $[\![nop]\!]=\lambda G_{<e,d>}\lambda d \lambda x.[G(x) \geq d]$

        \item \label{gao_nop_LE_b} 
        $[\![tall]\!]=Height(x)$

        \item \label{gao_nop_LE_c} 
        $\begin{aligned}[t]
            [\![nop \enspace tall]\!] &= [\![nop]\!]([\![tall]\!]) \\
            &= \lambda d \lambda x.[Height(x) \geq d]
        \end{aligned}$

    \end{enumerate}
\end{enumerate}

Although the lexical result of null operator and adjectives in \ref{gao_nop_LE_c} looks exactly same with traditional lexical entry of gradable adjectives in \ref{tallLE_a}, which makes this null operator seems redundant. But this "null operator + gradable adjective" structure resolve the problem mentioned in \ref{old_school_def_problem}. Actually, this design separates the "individual property function" from "greater equal meaning", the former's owner is gradable function, and latter's owner is null operator. It is null operator who does not select for \textit{short} when measure phrase is \textit{2 meters}. And, this idiosyncratic property of null operator is language-specific. The advantage of this kind of structure is leaving language-specific problem to null operator and keeping gradable adjectives away form any idiosyncratic language-specific properties.

Next questions are, whether this null operator has a specific phonetic expression in any language and where its position is in syntactic structure. 

For the first question, Svenonius does a deep investigation about Icelandic and Norwegian, which shows that a phonetic word of this null operator does not exist in Norwegian but does exist in Icelandic. In Icelandic, \ref{Icelandic_example} gives three different ways to express the same meaning. In \ref{Icelandic_example_10}, the word \textit{Hversu} is mapping to two English words \textit{how.much}, but actually \textit{Hversu} only used as a degree operator, and it does not bear the function of manner adverbial. In \ref{Icelandic_example_a}, speaker can omit word \textit{Hversu} and \colorbox{yellow}{front predicate} to express the same meaning. Also, in \ref{Icelandic_example_b}, a word \textit{Hvað} can be placed at the beginning of the sentence, and keep all other morphemes in their original position. What should be noticed is that the Icelandic word \textit{Hvað} does not have real meaning, it is just a phonologically placeholder, which exactly is an evidence of the existence of null operator. For the latter question, Svenonius gives syntactic structures of \ref{Icelandic_example} respectively as shown in \ref{Icelandic_example_LE}. From Svenonius' perspective, null operator and the word \textit{Hvað} show the same locality conditions which is same with famous \textit{wh}-movement. 

\begin{enumerate}[resume]
    \item \label{Icelandic_example}
    
    \begin{enumerate}[ref=(\arabic{enumi}\alph*)]
        \item \label{Icelandic_example_10}
        Hversu gammall ertu? \\
        \textit{how.much old are.you} \\
        how old are you?

        \item \label{Icelandic_example_a}
        er du gammel? \\
        \textit{are you old} \\
        how old are you?

        \item \label{Icelandic_example_b}
        Hvað ertu gammall? \\
        \textit{null are you old} \\
        how old are you?

    \end{enumerate}   
    
\end{enumerate}

\begin{enumerate}[resume]
    \item \label{Icelandic_example_LE}
    
    \begin{enumerate}[ref=(\arabic{enumi}\alph*)]
        \item $[_{CP}Nop_1 \enspace er_2[_{IP}du_3 \enspace t_2[_{VP} t_2[_{AP} t_3 \enspace t_1 \enspace gammel]]]]$
        
        \item $[_{CP}Hva$ð$_1 \enspace er_2[_{IP} \enspace -tu_3 \enspace t_2 [_{VP}t_2[_{AP}t_3 \enspace t_1 \enspace gammel]]]]$
        
    \end{enumerate}   
    
\end{enumerate}

The $<e,d>$ type gradable adjectives seems a perfect definition. But I have to point out that, both lexical entries in \ref{tallLE} may cause a mismatch between the meaning given by lexical calculation and meaning given by language common sense. 

Since there are serval kinds of ways to build a degree construction, such as positive form, comparative form and superlative form, a lexical meaning should have ability to give a interpretation of all these constructions. \ref{tallLE_a} gives a positive form example, and it's lexical entry calculation is shown in \ref{john_gao_2_mi_LE}, who uses traditional gradable adjectives lexical type in \ref{tallLE_a}. To resolve $\lambda$ reduction, the lexical entry of \textit{tall} needs a measure phrase with lexical entry $d$ and an individual $x$ , which are respectively \textit{2 meters} and \textit{John}. \textit{2 meters} changes the semantic type of \textit{tall} from $<d,<e,t>>$ to $<e,t>$, and \textit{John} is input as type $e$ which makes final result to type $t$. The result of lexical entry calculation tells a truth that John's height is not only equal to 2 meters precisely, but also has a possibility to greater than 2 meters. But actually we do know that sentence in \ref{john_gao_2_mi} means John's height is 2 meters accurately, which is mismatch with the result of semantic calculation. The reason why this mistake is made is that the partial ordering relation of degree is encoded in gradable adjectives, \colorbox{yellow}{and there is no way to resolve this "greater equal"}. \ref{tallLE_b} will also lead this problem, since the lexical entry of the combination of null operator and gradable adjective is same with the lexical entry shown in \ref{tallLE_a}.

\begin{enumerate}[resume]
    \item \label{john_gao_2_mi} Jhon is 2 meters tall.
\end{enumerate}

\begin{enumerate}[resume]
    \item \label{john_gao_2_mi_LE}
    
    \begin{enumerate}[ref=(\arabic{enumi}\alph*)]
    \item \label{john_gao_2_mi_LE_a} 
    $[\![tall]\!] = \lambda d \lambda x.[Height(x) \geq d]$
    
    \item \label{john_gao_2_mi_LE_b} 
    $\begin{aligned}[t]
        [\![2 \enspace meters \enspace tall]\!] &= [\![tall]\!](2m) \\
        &= \lambda x.[Height(x) \geq 2m]
    \end{aligned}$
    
    \item \label{john_gao_2_mi_LE_c} 
    $\begin{aligned}[t]
        [\![John \enspace is \enspace 2 \enspace meters \enspace tall]\!] &= [\![tall \enspace 2m]\!](John) \\
        &= Height(John) \geq 2m
    \end{aligned}$
    
    \end{enumerate}
\end{enumerate}

To fix the problem mentioned above, I am going to modify the lexical entry of null operator, and for convenience, I will give null operator a simple name $\mu$. New lexical entry of $\mu$ is shown in \ref{mu_LE_1}, and the result lexical entry after the combination with gradable adjective \textit{tall} is shown in \ref{mu_LE_2}.

\begin{enumerate}[resume]
    \item \label{mu_LE}
    
    \begin{enumerate}[ref=(\arabic{enumi}\alph*)]
        \item \label{mu_LE_1}
        $[\![\mu]\!] =  \lambda G_{<e,d>}\lambda d \lambda x.[G(x) = d]$
        \item \label{mu_LE_2}
        $[\![\mu \enspace tall]\!] = \lambda d \lambda x.[Height(x) = d]$
    \end{enumerate}
\end{enumerate}

Based on this "equal meaning" lexical entry of gradable adjectives, we can solve the problem which "greater equal" meaning lexical entry can not. Applying this $\mu$'s lexical entry to Mandarin, I give a example in \ref{john_gao_2_mi_mu}, which is a positive form degree construction. The lexical calculation is shown in \ref{john_gao_2_mi_mu_LE}. The procedure of calculation is generally same with \ref{john_gao_2_mi_LE}, but because the $\mu$ has "equal meaning", the result was lead to a correct way which gives a truth that John's height is exactly 2 meters, and that is exactly what we expect to get.

\begin{enumerate}[resume]
    \item \label{john_gao_2_mi_mu} 
    Jhon gao 2 mi. \\
    \textit{Jhon tall 2 meters.} \\
    Jhon is 2 meters tall.
\end{enumerate}

\begin{enumerate}[resume]
    \item \label{john_gao_2_mi_mu_LE}
    
    \begin{enumerate}[ref=(\arabic{enumi}\alph*)]
    \item
    $[\![\mu \enspace gao]\!] = \lambda d \lambda x.[Height(x) = d]$
    
    \item
    $\begin{aligned}[t]
        [\![\mu \enspace gao \enspace 2 \enspace mi]\!] &= [\![\mu \enspace gao]\!](2m) \\
        &= \lambda x.[Height(x) = 2m]
    \end{aligned}$
    
    \item
    $\begin{aligned}[t]
        [\![John \enspace \mu \enspace gao \enspace 2 \enspace mi]\!] &= [\![\mu \enspace gao \enspace 2 \enspace mi]\!](John) \\
        &= Height(John) = 2m
    \end{aligned}$
    
    \end{enumerate}
\end{enumerate}


\subsection{Application in Mandarin}

In last section, I introduce two traditional lexical entries of gradable adjectives, and illustrate their flaws. At the end of last section, I propose my improvement of the lexical entry of gradable adjectives. In this section, I will show how this new definition work in Mandarin degree construction.

\subsubsection{Positive form}

A simple example of positive form was shown in last section in \ref{john_gao_2_mi_mu}. In Mandarin, some decorations can be added in positive form like "\textit{hen}(very)", "\textit{youdian}(a little)", 


\subsubsection{Comparative form}


Applying this $\mu$'s lexical entry to Mandarin, I give two examples in \ref{john_gao_2_mi_nop}. In \ref{john_gao_2_mi_nop_positive}, the degree construction is positive form, so the null operator does not exist, gradable adjective \textit{gao} composite with measure phrase \textit{2mi} directly, which makes final lexical entry calculation result is $Height(John)=2mi$. That gives a truth that John's height is exactly 2 meters, which is match with what we expect. When construction comes to comparative form like \ref{john_gao_2_mi_nop_comparative} with lexical entry calculation in \ref{john_gao_2_mi_nop_comparative_LE}. There will be a null operator who should combine with \textit{gao} first, and changes the meaning of \textit{gao} from "equal meaning" to "greater equal meaning". Then, the lexical entry of important comparative morpheme \textit{bi} is shown in \ref{john_gao_2_mi_nop_comparative_LE_bi}.

\begin{enumerate}[resume]
    \item \label{john_gao_2_mi_nop}
    
    \begin{enumerate}[ref=(\arabic{enumi}\alph*)]
        \item \label{john_gao_2_mi_nop_positive}
        Jhon ($\mu$) gao 2 mi.  \\
        \textit{Jhon ($\mu$) tall 2 meters.}    \\
        Jhon is 2 meters ($\mu$) tall.

        \item \label{john_gao_2_mi_nop_comparative}
        Jhon bi Marry (nop) gao. \\  
        \textit{Job er Marry (nop) tall.}    \\
        Jhon is taller than Marry.

    \end{enumerate}
    
\end{enumerate}

\begin{enumerate}[resume]
    \item \label{john_gao_2_mi_nop_comparative_LE}
    \begin{enumerate}[ref=(\arabic{enumi}\alph*)]
        \item $[\![tall]\!]=Height(x)=d$
        \item $[\![nop]\!]=\lambda G\lambda d \lambda x.G(x) \geq d$
        \item 
        $\begin{aligned}[t]
            [\![nop \enspace tall]\!] &= [\![nop]\!]([\![tall]\!]) \\
            &= \lambda d \lambda x.Height(x) \geq d
        \end{aligned}$ \label{john_gao_2_mi_nop_comparative_LE_bi}
        \item $[\![bi]\!]= \lambda G_{<d,<e,t>>} \lambda y \lambda x . \exists \delta > 0 \enspace s.t.[G(x,d_x^{\prime}) - G(y,d_y^{\prime})>\delta]$
        \item
        $\begin{aligned}[t]
            [\![bi \enspace nop \enspace gao]\!] &= [\![bi]\!]([\![nop \enspace tall]\!]) \\
            &= \lambda y \lambda x . \exists \delta > 0 \enspace s.t.[Height(x,d_x^{\prime}) - Height(y,d_y^{\prime})>\delta]
        \end{aligned}$
        \item
        $\begin{aligned}[t]
            [\![John \enspace bi \enspace Marry \enspace nop \enspace gao]\!] 
            &= [\![bi \enspace nop \enspace gao]\!]([\![John]\!])([\![Marry]\!]) \\
            &= \exists \delta > 0 \enspace s.t.[Height(John,d_{John}^{\prime}) \\ 
            & \qquad - Height(Marry,d_{Marry}^{\prime})>\delta]
        \end{aligned}$
    \end{enumerate}
\end{enumerate}

The second "equal meaning" lexical entry in \ref{tallLE_b} looks like a perfect solution who resolve the meaning-mismatch problem via null operator. But, I have to point out that, this "equal meaning" solution still causes some mismatch mistakes. To illustrate this conclusion more clear, we have to find out the lexical entry of comparative morpheme first. Back to example \ref{john_gao_2_mi_nop_comparative}









such as following example in \ref{equal_mistake_example}, with lexical entry calculation shown in \ref{equal_mistake_example_LE}.

\begin{enumerate}[resume]
    \item \label{equal_mistake_example}
    Jhon bi Marry (nop) gao 2 mi. \\
    \textit{Job er Marry (nop) tall 2 meters.}    \\
    Jhon is 2 meters taller than Marry.
\end{enumerate}

\begin{enumerate}[resume]
    \item \label{equal_mistake_example_LE}
    \begin{enumerate}[ref=(\arabic{enumi}\alph*)]
        \item $[\![gao]\!]=Height(x)=d$
        \item $[\![nop \enspace tall]\!] = \lambda d \lambda x.Height(x) \geq d$
        \item $[\![nop \enspace tall \enspace 2mi]\!] \lambda x.Height(x) \geq 2mi$ 
        \item $[\![biP]\!]=\lambda G \lambda x \lambda y \lambda d. \exists d_1 \exists d_2 [(G(x) \geq d_1) \land (G(y) \geq d_2) \land (d_1-d_2 \geq 2m) ]$
    \end{enumerate}
\end{enumerate} 

\section{Complex problem in Mandarin}


\section{Conclusion}

\newpage

\printbibliography

\end{document}
