\documentclass{ctexart}

\usepackage{geometry}
\usepackage{ctex}
\usepackage{color}
\usepackage{float}
\usepackage{soul}
\usepackage{amsmath}
\usepackage[british]{babel}
\usepackage[utf8]{inputenc}
\usepackage{epstopdf}
\usepackage{csquotes}
\usepackage[hidelinks]{hyperref}
\usepackage[T1]{fontenc}
\usepackage{enumitem}
\usepackage{caption} 
\usepackage{graphicx}  % Required for inserting images
\usepackage[
    style=apa,
    backend=biber,
    sortcites=true,
    sorting=nyt,
%    isbn=false,
%    url=false,
%    doi=false,
%    eprint=false,
    hyperref=false,
    backref=false,
%    firstinits=false,
]{biblatex}

% section title format
\ctexset{
    section={
        name={,.},
        format+ = \raggedright, % 一级标题左对齐
        aftername = \hspace{4pt}
    },
    subsection={
        titleformat+ = \textit, % 标题内容斜体
        aftername = \hspace{2pt}
    },
    subsubsection={
        titleformat+ = \textit, % 标题内容斜体
        aftername = \hspace{2pt}
    },
}

% figure caption
\captionsetup[figure]{
    labelfont = {it}, % 斜体Figure
    labelsep = space, % 去掉冒号
}

% maps apacite commands to biblatex commands
\let \citeNP \cite
\let \citeA \textcite
\let \cite \parencite

\bibliography{reference}

% 1英寸边界
\geometry{a4paper, right=1in, left=1in, top=1in, bottom=1in} 

% 页码放下面
\pagestyle{plain} 

% 1.5行间距
\linespread{1.5} 

% title
\title{Semantics and Syntax of Degree Construction in and Mandarin}
\date{\vspace{-10ex}}

% enumerate列表格式
\setlist[enumerate,1]{
    resume,
    label=(\arabic*)
    }

\setlist[enumerate,2]{
    leftmargin=*,
    labelindent=16pt,
    label=(\alph*),
    ref=(\arabic{enumi}\alph*),
    }

%%%%%%%%%%%%%%%%%%%%%%%%%%%%%%%%%%%%%%%%%%%%%%%%%%%

\begin{document}

\thispagestyle{empty} % 这一页清空

\begin{center}

Name: $<<$* Zhang Yi Feng *$>>$ \\
Student ID: $<<$* 1155166448 *$>>$

\end{center}

{\let\newpage\relax\maketitle}

% 简介一下比较,给出中英例子,中文比较级分类
% 主要要说明清楚,positive comparative 这几个分类

\section{INTRODUCTION}

\setcounter{page}{1}

\noindent
Gradability is an essential concept in the studies of adjectives, which classifies adjectives into two major classes, non-gradable adjectives, like \textit{British}, etc. and gradable adjectives, like \textit{tall}, \textit{long}, etc. Gradable adjectives are important role to build degree comparative constructions in a language. In this section, this paper will go through a batch of examples of comparatives and categorize them by their different structures, and this classification will lead semantic and syntactic analysis in next sections. 

From traditional view, ingredients of a comparative construction are, a comparee, a standard, a comparative marker and a gradable predicate \cite{guo2012}. And from this this paper's perspective, in semantic, a comparative construction can be departed to a comparee, a standard, a gradable predicate, and a optional differential phrase; in syntactic, there are two more ingredients, comparative marker and comparative morpheme.

A simplest example in English is shown in \ref{degree_construction_example}, in this sentence, \textit{John} is comparee, \textit{Mary} is standard, word \textit{than} is a comparative marker whose job is introduce strict partial relation meaning ``greater than'' \hl{ref here}, and suffix -\textit{er} is another comparative marker whose function is still controversial. Some researchers believe that -\textit{er} is just a word in specifier position without crucial function \cite{von1984a,heim1985,bhatt2004,rullmann1995}, another kind of view treat this word as a head of degree phrase (DegP) \cite{bierwisch1989,corver1990,corver1993,corver1997a,kennedy1997,grano2012}. Different form English, Mandarin is considered as a single mark language in comparative expression \cite{bobaljik2012,grano2012}. In most of cases, this mark is morpheme \textit{bi}(`than') such as example in \ref{degree_construction_example_Mandarin}, and there is no inflection of adjectives in Mandarin like -\textit{er} in English \cite{guo2012}. 

\begin{enumerate}
    \item \label{degree_construction_example}
    John is taller than Mary.
\end{enumerate}

\begin{enumerate}
    \item \label{degree_construction_example_Mandarin}
    John bi Mary gao.  \\
    John than Mary tall. \\
    `John is taller than Mary.'
\end{enumerate}

Differential phrase (DiffP) is an important part of comparative structure, whose job is giving the differential scale of context individuals. A comparative structure with differential phrase in English is shown in \ref{degree_construction_dp_example}, and Mandarin version is shown in \ref{degree_construction_dp_example_Mandarin}.

\begin{enumerate}
    \item \label{degree_construction_dp_example}
    John is 2 meters taller than Mary.
\end{enumerate}

\begin{enumerate}
    \item \label{degree_construction_dp_example_Mandarin}
    John bi Mary gao 2 mi.  \\
    John than Mary tall 2 meters. \\
    `John is 2 meters taller than Mary.'
\end{enumerate}

The differential phrase in examples above are specific value differential phrase, which gives a accurately value of scale about gradable adjective. Another type of differential phrase gives vague value of scale. When this kind of differential phrase appears in comparative meaning, there is always a standard scale existed in context, and this vague value is either bigger than the standard value, or smaller than the standard value. Like example in \ref{dp_big_vague_example}, morpheme \textit{hen duo}(`much') is a big value vague differential phrase, which means, John is not only taller than Mary, but also, the difference of their height is lager than a standard value. This standard value is given in context, it is a consensus between speakers. Similarly, a small value vague differential phrase is shown in \ref{dp_small_vague_example}. In Mandarin, vague differential phrase can be complex, many researchers \cite{lin2014,li2015} do deeply investigation on it, \ref{dp_value_big_vague_example} and \ref{dp_value_small_vague_example} show such type of differential phrase which appears as ``vague prefix + accurately value'' or ``accurately + vague suffix''.

\begin{enumerate}
    \item
    \begin{enumerate}
        \item \label{dp_big_vague_example}
        John bi Mary gao hen duo.\\
        John than Mary tall much. \\
        `John is much taller than Mary.'

        \item \label{dp_small_vague_example}
        John bi Mary gao yi dian.\\
        John than Mary tall a little.\\
        `John is a little taller than Mary.'

    \end{enumerate}
\end{enumerate}

\begin{enumerate}
    \item
    \begin{enumerate}
        \item \label{dp_value_big_vague_example}
        John bi Mary gao liang mi duo.\\
        John than Mary tall 2 meters more. \\
        `John is more than 2 meters taller than Mary.'

        \item \label{dp_value_small_vague_example}
        John bi Mary gao bu dao liang mi.\\
        John than Mary tall less 2 meters. \\
        `John is less than 2 meters taller than Mary.'
    \end{enumerate}
\end{enumerate}

In comparative structure built by gradable adjectives, sometimes, the standard is not a specific individual. This paper propose that there are three categories of standard, which are ``single individual standard'', ``individual set standard'' and ``specific value standard''. All examples mentioned above is single individual comparison. \ref{specific_value_comparison_example} gives a example of specific value comparison, in which standard is not a individual. \ref{individual_set_comparison_example} shows a example of individual set comparison.

\begin{enumerate}
    \item
    \begin{enumerate}
        \item \label{specific_value_comparison_example}
        John bi 2 mi gao.\\
        John than 2 meters tall. \\
        `John is taller than 2 meters.'

        \item \label{individual_set_comparison_example}
        zai yi ban, John zui gao.\\
        in one class, John most tall.\\
        `John is tallest in class one.'

    \end{enumerate}
\end{enumerate}

Then we propose a classification of degree structure into two types, assignable meaning and comparative meaning.

\begin{itemize}

    \item[1.] \textbf{Assignable meaning}: a degree structure is assignable meaning if and only if the function of gradable adjective is assignment. Under assignable meaning, there is no more subtypes, \ref{assignable_meaning_example} shows a example, where gradable adjective assigns a accurately value \textit{2 mi}(`2 meters') to another individual's height. When a scale-related group of words appears in assignable meaning, this paper do not call it differential phrase, but measure phrase (MP). Many researchers do not separate this two concepts very clear. Under our discussion, differential phrase appears in assignable meaning, measure phrase appears in comparative meaning. 
    
    \begin{enumerate}
        \item \label{assignable_meaning_example}
        John gao 2 mi.\\
        John tall 2 meters. \\
        `John is 2 meters tall.'
    \end{enumerate}
    
    \item[2.] \textbf{Comparative meaning}: a degree structure is assignable meaning if and only if the function of gradable adjective is comparison. Under comparative meaning, there are three subtypes, positivity, superiority and equality. 
    
    \begin{itemize}

        \item[i.] \textbf{Comparative of positivity}: a comparative meaning degree structure is positivity if and only if the standard is implicit. \ref{positivity_example} gives a example of positivity, this utterance tells a truth that, the value of John's height is greater than a standard which exists under the context between speakers. 
        
        \begin{enumerate}
            \item \label{positivity_example}
            John hen gao. \\
            John very tall. \\
            `John is very tall.'
        \end{enumerate}
        
        \item[ii.] \textbf{Comparative of superiority}: a comparative meaning degree structure is superiority if and only if the standard is explicit and the difference between comparee and standard is strictly greater than zero. \ref{degree_construction_example_Mandarin} is a example of comparative of superiority, which is basic type of degree structure. what should be noticed is example in \ref{individual_set_comparison_example}, which is also belong to comparative meaning type and superiority subtype. This degree structure belongs to comparative meaning because the gradable adjective here is still bear comparison job, which compare height between John and an individual set. And the example is comparative of superiority on account of that this superlative expression gives a truth that the the value of John's heigh is strictly greater than any of individual set.
        
        \item[iii.] \textbf{Comparative of equality}: a comparative meaning degree structure is superiority if and only if the standard is explicit and the difference between comparee and standard is equal zero. \ref{equality_example} shows a example of comparative of equality, which gives a expression that the value of John's height is same with the value of Mary's height. Here may have some controversies, someone may think that, \ref{equality_example} actually assigns the value of Mary's height to John's height, which will lead the function of gradable adjective turn to assignment \cite{guo2012}. This paper argues that, the height is kind of inner property of an individual, so we can not assign one's height to another. On the opposite, the essence of equality is to express a truth that, the difference between tow individuals height is zero, thus the function of gradable adjective in equative form is comparison rather than assignment.
        
        \begin{enumerate}
            \item \label{equality_example}
            John he Mary yi yang gao.\\
            John and Mary same tall.\\
            `John is as tall as Mary.'
        \end{enumerate}

    \end{itemize}   

\end{itemize}

Next phenomena should be noticed is the visibility of the standard. See example in \ref{positivity_example}, in traditional research \hl{ref here}, this example is classified in assignable meaning, but by this paper's approach, this example is comparative meaning because the gradable adjective bears the comparison function, rather than the assignment function. So the problem raises up which is, there is no standard to comparison. Actually, here do exist a compare target under this context. The morpheme \textit{very tall}, means that there is a standard height between speakers, and the value of John's height is greater than that standard. So this ``specific value standard'' is actually a implicit standard.

Up to now, it is time to summarize all kinds classification mentioned above. First of all, a simple degree structure built by gradable adjectives can be classified into two categories, comparative meaning and assignable meaning. If gradable adjectives bear value assignment function, sentence can be seen as assignable meaning, and when gradable adjectives bear value comparison function, the degree structure can be seen as comparative meaning. And in comparative meaning, sentence can be built to positivity, superiority and equative. In superiority, there are three different angles to classified degree structure. 

\begin{itemize}
    \item[1.] The differential phrase is explicit or implicit.
    \item[2.] The standard is specific value or single individual or individual set.
    \item[3.] The standard is explicit or implicit.
\end{itemize}

For now, we give a clear classification to degree structures, and in next sections, we are going to make a deep discussion about the syntactic and semantic properties of those degree structures. In section 2, this paper will enumerate some former researchers' work about degree semantics, especially the lexical entry of gradable adjectives, the DegP structure and the DegP-shell structure. In section 3, this paper will based on former's achievement and the classification given above, illustrate a new syntactic and semantic analysis in Mandarin degree structure.

\section{FORMER DEGREE SEMANTICS RESEARCH}

\noindent
First founder of analysis on gradable adjectives is Cresswell \cite{cresswell1976}, who creates a third primitive semantic type $d$ besides the two primitive semantic types $e$ and $t$ in classical semantics Degree semanticists also identify three major parts of gradable adjectives:

\begin{itemize}
    \item[1.] A measure function $G$, mapping the target $x$ onto the abstract dimension for measurement characteristic of the gradable adjective \cite{bartsch1974}.
    \item[2.] The total ordering relation $\geq$, which makes the set of scales corresponding to the abstract dimension ordered in pairs of a same direction.
    \item[3.] The degree variable, indicating the value of $G(x)$. A gradable adjective of predictive use is analysed as a two-place predicate with the individual and the degree as its arguments.
\end{itemize}

After this, lots of research of degree sematic are raised up in past decades. Generally, there are two directions to discuss degree semantics, a syntactic way and semantic way. In syntax, scholars talk about the syntactic structure of degree phrase(DegP) and adjective phrase(AP), and serval classical structure came up in years. In semantic, scholars discuss the lexical entry of gradable adjectives and other morphemes appeared in degree construction, make lexical calculation, and try to make the application of their definition as large as possible.

\subsection{Syntactic way}

At the beginning, most researchers believe degree phrase is a specifier of adjective phrase \cite{chomsky1977,selkirk2015,bresnan1973,heim2000}, this kind of structure shows in \ref{old_deg_structure}. The main problem of this structure is that, standard as complement of the degree head will cause different degree heads lead different kinds of complements. In example \ref{old_deg_structure_example} shows this conflict, morpheme \textit{er} selects \textit{than Mary} as its complement, and \textit{est} selects \textit{all students} as its complement. But \textit{est} can not select \textit{than Mary}, which will cause grammar failure. The other problem of this kind of syntactic structure is that, a movement exists in this structure which makes gradable adjective moves from head of AP to AP's specifier phrase's head, aiming to combine with the degree head, shown in \ref{traditional_move}. This is not a acceptable movement.  

\begin{enumerate}
    \item \label{old_deg_structure} $[_{AP}[_{DegP}Deg[Standard]][_{A}...]]$
\end{enumerate}

\begin{enumerate}
    \item \label{old_deg_structure_example}
    \begin{enumerate}
        \item John is taller than Mary.
        \item John is tallest of all students.
    \end{enumerate}
\end{enumerate}

\begin{enumerate}
    \item \label{traditional_move}
\end{enumerate}

\begin{figure}[H]
    \centering
    \includegraphics[width=0.4\textwidth]{pic/traditional_move.png}
    \begin{caption}
        \\ \vspace{-1.1ex}
        Adjective movement when DegP as specifier of AP.
    \end{caption}
\end{figure}

Abney argues that degree phrase is a functional projection which is above the adjective phrase \cite{abney1987}, and a lot of researchers follow it \cite{corver1993,zwarts1992}. This structure treats DegP as a natural extended projection of the gradable adjectives \cite{grimshaw2005}, and as a functional projection, selecting AP as its complement. This theory is called DegP hypothesis by later scholars. In DegP hypothesis, the general structure is described as followed. The XP in \ref{abney_deg_p} is different in different situation, it may quantifier phrase, or differential phrase, or measure phrase. Three examples and their syntactic structures shown in \ref{abney_deg_p_example}.

\begin{enumerate}
    \item \label{abney_deg_p} $[_{DegP} XP [_{Deg^{\prime}}Deg^0[_{AP}...]]] $
\end{enumerate}

\begin{enumerate}
    \item \label{abney_deg_p_example}
    \begin{enumerate}
        \item \label{abney_deg_p_example_1}
        much too tall.\\
        $[_{DegP} [_{QP} much]] [_{Deg^{\prime}}Deg^0 \enspace too [_{AP}tall]] $
        \item \label{abney_deg_p_example_2}
        two meters as tall.\\
        $[_{DegP} [_{MP} two \enspace meters]] [_{Deg^{\prime}}Deg^0 \enspace as [_{AP}tall]] $
        \item \label{abney_deg_p_example_3}
        two meters taller.\\
        $[_{DegP} [_{DiffP} two \enspace meters]] [_{Deg^{\prime}}Deg^0 \enspace er [_{AP}tall]] $

    \end{enumerate}
\end{enumerate}

DegP hypothesis fixes the problems mentioned above. For the first problem, degree head's complement becomes AP, and the comparee become AP's specifier. For the second problem, form \ref{degp_hypothesis}, we can tell that the gradable adjective moves form AP's head to DegP's head. The head of DegP, giving the comparative morpheme \textit{er} as the typical one, takes a ``than phrase'' (thanP) as its complement and a differential phrase as its specifier. Depending on various categories of complement in thanP as well as the overt appearance, such as \textit{2 meters}, or covert appearance of differential phrase. \textit{er} has kinds of semantic variants. First, \textit{er} can take a direct degree expression, such as than \textit{2 meters}, which denotes a degree argument typed $d$. Second, a comparative clause, such as \textit{than Mary is tall}, which denotes a property of degree argument, typed $<d,t>$ because according to the view of Chomsky, the comparative clause \textit{than Mary is tall} owns a deep structure looking like ``\textit{than} $how_i$ \textit{Mary is} $t_i$ \textit{tall}'' which undergoes \textit{wh}-movement, leaving the trace $t_i$ denoting a degree variable bound by $\lambda$-operator \cite{chomsky1977}. Third, a bare NP, such as \textit{than Mary}, regarded as a deletion from the full comparative clause \textit{than Mary is tall}, also denotes a property of degree argument typed $<d,t>$. Examples illustrated in \ref{er_var_example} are possible semantic variants of \textit{er}.

\begin{enumerate}
    \item \label{degp_hypothesis}
\end{enumerate}

\begin{figure}[H]
    \centering
    \includegraphics[width=0.4\textwidth]{pic/degP_hy.png}
    \begin{caption}
        \\ \vspace{-1.1ex}
        Adjective movement in DegP hypothesis.
    \end{caption}
\end{figure}

\begin{enumerate}
    \item \label{er_var_example}
    \begin{enumerate}
        \item \label{er_var_1} John is taller than 6 feet.
        \item \label{er_var_2} John is taller than Mary.
        \item \label{er_var_3} John is taller than Mary is.
        \item \label{er_var_4} John is 6 inches taller than Mary is.
    \end{enumerate}

\end{enumerate}

After basic DegP hypothesis, there are serval modified versions raised up by later researchers. The most important modification of DegP hypothesis is the Larson's two layers DegP-shell structure \cite{larson1991}, and this kind of two layers shelled structure is the most famous structure in degree semantic these years \cite{grano2012,guo2012,xiang2005,fabregas2020}. In Larson's research, he propose a DegP-shell based on VP-shell structure, which is shown in \ref{deg_p_shell}. 

\begin{enumerate}
    \item \label{deg_p_shell} $[_{DegP} [_{Deg^{\prime}} Deg^0 [_{DegP} AP [_{Deg^{\prime}} Deg^0 [_{PP}...]]]]] $
\end{enumerate}

In this structure, the specifier of higher DegP is empty position, whose only job is to introduce lower DegP's head. So there are two movements need to be done. \ref{larson_structure} gives a movement structure of \textit{taller than Mary}. The first movement is comparative morpheme \textit{er} moving form lower DegP's head to higher DegP's head, where is empty position in original, second movement is gradable adjective moving form lower DegP's specifier to higher DegP's head, to combine with comparative morpheme \textit{er} moved here before.

\begin{enumerate}
    \item \label{larson_structure}
\end{enumerate}

\begin{figure}[H]
    \centering
    \includegraphics[width=0.5\textwidth]{pic/larson1.png}
    \begin{caption}
        \\ \vspace{-1.1ex}
        Movements in Larson's DegP-shell structure.
    \end{caption}
\end{figure}

In Mandarin degree semantic, Xiang gives his structure with AP as higher DegP's complement and lower DegP as AP's complement. Structure's layout is shown in \ref{xiang_structure} explaining the movement of \ref{xiang_structure_example}, illustrates that AP is sandwiched by two DegPs \cite{xiang2005}. This structure is adopted by Grano who does little adjustment to explain transitive degree structure in Mandarin \cite{grano2012}. 

\begin{enumerate}
    \item \label{xiang_structure_example}
    John bi Mary gao chu 2 limi.  \\
    John than Mary tall exceed 2 centimeters. \\
    `John is 2 centimeters taller than Mary.'
\end{enumerate}

\begin{enumerate}
    \item \label{xiang_structure}
\end{enumerate}

\begin{figure}[H]
    \centering
    \includegraphics[width=0.5\textwidth]{pic/xiang.png}
    \begin{caption}
        \\ \vspace{-1.1ex}
        Movements in Xiang's DegP-shell structure.
    \end{caption}
\end{figure}


\subsection{Semantic way}

The definition of lexical entry of gradable adjectives is literally important, because different definitions of adjectives' lexical entry always lead totally different results in semantics and syntax just like butterfly effect. Since Cresswell, the semantic type of gradable adjectives is largely debated. Von Stechow proposes a comprehensive constructive analysis of comparisons \cite{von1984a}, which becomes the so-called standard analysis later \cite{bale2011}. From his view, the meaning of gradable adjectives are interpreted as a measure function and an ordering relation. The measure function maps the individual to the dimension denoted by gradable adjectives and the ordering relation ensures that the scale of the individual exceeds the degree to compare. In generative grammar, a gradable adjective is the head of adjective phrase. It functions as a two-place predicate, with an individual typed $e$ and a degree typed $d$ as its two arguments. What deserves a note is it is von Stechow who in first seriously regards degrees denoted by symbol $d$ as one of the primitive semantic types and it is degree $d$ that captures the difference between gradable adjectives and non-gradable adjectives. Here the semantic type of gradable adjectives is manifested as $<d,<e,t>>$. Still giving \textit{tall} as an instance, the lexical entry of a gradable adjective is shown in \ref{von_stechow_LE}, in which height denotes the measure function encoded by \textit{tall} and $\geq$ denotes the ordering relation between the individual and the degree:

\begin{enumerate}
    \item \label{von_stechow_LE} $[\![tall]\!] = \lambda d \lambda x. [height(x) \geq d]$
\end{enumerate}

The analysis of DegP headed by a degree morpheme is much more complicated. The semantic type of DegP is $<d,<e,t>>$. Based on the approach in generative grammar, DegP lands at an adjunct position of AP in the deep structure and then undergoes quantifier raising in the logical form to a node above the original IP inside which DegP is initially located, with the trace left denoting a $d$ type argument. The motivation for this movement is that, according to von Stechow's analysis, DegP can be regarded as isomorphic to a generated quantifier phrase, an account under large debate afterwards.

According to type-driven computation in formal semantics, the gradable adjective typed $<d,<e,t>>$ first combines the trace of the semantic type $d$ which is left by DegP in the process of QR, then combines the subject in the matrix clause of the semantic type $e$, outputting a $t$ type proposition with a free degree variable. $\lambda$-abstraction turns this $t$ type open proposition into a property of degree typed $<d,t>$ which saturates DegP typed $<<d,t>,t>$. Finally, a $t$ type proposition is made out. And this is the procedure in von Stechow's analysis of how to derive a comparative construction.

Base on $<d,<e,t>>$ type gradable adjective and syntactic structure mentioned above, we can give the lexical entries in \ref{er_var_example}. The thanP takes a direct degree expression as complement in \ref{er_var_1}, a bare NP in \ref{er_var_2}, a comparative clause in \ref{er_var_3} and \ref{er_var_4}. Optional differential phrase only owns overt appearance in \ref{er_var_4}. Based on the analysis above, sematic type of \textit{er} in \ref{er_var_1} is $<d,<<d,t>,t>>$, in \ref{er_var_2} as well as \ref{er_var_3} is $<<d,t>,<<d,t>,t>>$, in \ref{er_var_4} is $<<d,t>,<d,<<d,t>,t>>>$.

Under this definition, gradable adjective \textit{tall} is considered as a relation of ``greater equal'', which is true when an individual $x$ as input and $x$'s height is at least as great as $d$. A simple application of this kind of lexical entry definition to \ref{old_school_def_example} is shown in \ref{old_school_def_example_LE}.

\begin{enumerate}
    \item \label{old_school_def_example} 2 meters tall.
\end{enumerate}

\begin{enumerate}
    \item \label{old_school_def_example_LE}
    \begin{enumerate}
        \item $[\![tall]\!]=\lambda d \lambda x.[Height(x) \geq d]$
        \item $[\![2 \enspace meters \enspace tall]\!]=\lambda x.[Height(x) \geq 2 \enspace meters]$
    \end{enumerate}
\end{enumerate}

Actually, lexical entry definition like \ref{von_stechow_LE} remains a problem with corresponding examples shown in \ref{old_school_def_problem}. In English, sometimes measure phrase can not directly combine with negative-pole adjectives. We can say someone is \textit{2 meters tall}, but can not say someone is \textit{2 meters short}. But situation changes when suffix \textit{er} shows up in \ref{old_school_def_problem_correct}, \textit{2 meters shorter} is a correct usage in English. Besides, this phenomenon is also a language-specific problem, in Japanese, we can not combine measure phrase even with \textit{segatakai}(`tall'). Traditional lexical entry of gradable adjectives has no ability to explain this phenomenon.

\begin{enumerate}

    \item \label{old_school_def_problem}
    
    \begin{enumerate}
        
        \item 2 meters tall. 10 years old.
        \item *2 meters short. *10 years young.
        \item \label{old_school_def_problem_correct} 2 meters shorter. 10 years younger.

    \end{enumerate}

\end{enumerate}

Based on the problem mentioned above, there is another group of researchers propose that the gradable adjectives' lexical entry should not encode the partial ordering relation, instead, the gradable adjectives should reveal the original property of an individual, which means \textit{tall} should simply illustrate the height of an individual. Thus the lexical entry of gradable adjectives should looks like \ref{tallLE_b}, in which there is just a measure function. 

\begin{enumerate}
    \item \label{tallLE_b} 
    $[\![tall]\!]=Height(x)=d$
\end{enumerate}

Under this assumption, the partial ordering relation still needs a place to be introduced, if not, there will be a lexical entry type mismatch. The type of measure function \textit{tall} is $<e,d>$, but measure phrase is type $d$, so gradable adjectives are no way to composite with measure phrase on account of type-theoretic. To resolve this problem, Svenonius \cite{svenonius2006} claims that there is a null operator whose semantic function is linking the lexical entry of gradable adjectives and the lexical entry of measure phrase, and syntactic function is to introduce a degree argument and bear the mission of introducing the ``greater equal'' meaning. The denotation of this null operator is spelled out in \ref{gao_nop_LE}, and the \ref{gao_nop_LE_c} shows the composition of null operator and gradable adjective, which is same with \ref{von_stechow_LE}. The lexical entry of \ref{gao_nop_LE_c} is $<d,<e,t>>$, which can combine with measure phrase with out any type conflict.

\begin{enumerate}
    \item \label{gao_nop_LE}

    \begin{enumerate}
        \item \label{gao_nop_LE_a} 
        $[\![nop]\!]=\lambda G_{<e,d>}\lambda d \lambda x.[G(x) \geq d]$

        \item \label{gao_nop_LE_b} 
        $[\![tall]\!]=Height(x)$

        \item \label{gao_nop_LE_c} 
        $\begin{aligned}[t]
            [\![nop \enspace tall]\!] &= [\![nop]\!]([\![tall]\!]) \\
            &= \lambda d \lambda x.[Height(x) \geq d]
        \end{aligned}$

    \end{enumerate}
\end{enumerate}

Although the lexical result of null operator and adjectives in \ref{gao_nop_LE_c} looks exactly same with traditional lexical entry of gradable adjectives in \ref{von_stechow_LE}, which makes this null operator seems redundant. But this ``null operator + gradable adjective'' structure resolve the problem mentioned in \ref{old_school_def_problem}. Actually, this design separates the ``individual property function'' from ``greater equal meaning'', the former's owner is gradable function, and latter's owner is null operator. It is null operator who does not select for \textit{short} when measure phrase is \textit{2 meters}. And, this idiosyncratic property of null operator is language-specific. The advantage of this kind of structure is leaving language-specific problem to null operator and keeping gradable adjectives away form any idiosyncratic language-specific properties.

Next questions are, whether this null operator has a specific phonetic expression in any language and where its position is in syntactic structure. 

For the first question, Svenonius does a deep investigation about Icelandic and Norwegian, which shows that a phonetic word of this null operator does not exist in Norwegian but does exist in Icelandic. In Icelandic, \ref{Icelandic_example} gives three different ways to express the same meaning. In \ref{Icelandic_example_10}, the word \textit{Hversu} is mapping to two English words \textit{how.much}, but actually \textit{Hversu} only used as a degree operator, and it does not bear the function of manner adverbial. In \ref{Icelandic_example_a}, speaker can omit word \textit{Hversu} and front the predicate to express the same meaning. Also, in \ref{Icelandic_example_b}, a word \textit{Hvað} can be placed at the beginning of the sentence, and keep all other morphemes in their original position. What should be noticed is that the Icelandic word \textit{Hvað} does not have real meaning, it is just a phonologically placeholder, which exactly is an evidence of the existence of null operator. For the latter question, Svenonius gives syntactic structures of \ref{Icelandic_example} respectively as shown in \ref{Icelandic_example_LE}. From Svenonius' perspective, null operator and the word \textit{Hvað} show the same locality conditions which is same with famous \textit{wh}-movement. 

\begin{enumerate}
    \item \label{Icelandic_example}
    
    \begin{enumerate}
        \item \label{Icelandic_example_10}
        Hversu gammall ertu? \\
        how.much old are.you \\
        `how old are you?'

        \item \label{Icelandic_example_a}
        er du gammel? \\
        are you old \\
        `how old are you?'

        \item \label{Icelandic_example_b}
        Hvað ertu gammall? \\
        null are you old \\
        `how old are you?'

    \end{enumerate}   
    
\end{enumerate}

\begin{enumerate}
    \item \label{Icelandic_example_LE}
    
    \begin{enumerate}
        \item $[_{CP}Nop_1 \enspace er_2[_{IP}du_3 \enspace t_2[_{VP} t_2[_{AP} t_3 \enspace t_1 \enspace gammel]]]]$
        
        \item $[_{CP}Hva$ð$_1 \enspace er_2[_{IP} \enspace -tu_3 \enspace t_2 [_{VP}t_2[_{AP}t_3 \enspace t_1 \enspace gammel]]]]$
        
    \end{enumerate}   
    
\end{enumerate}

The $<e,d>$ type gradable adjectives seems a perfect definition. But I have to point out that, both lexical entries in \ref{von_stechow_LE} and \ref{tallLE_b} may cause a mismatch between the meaning given by lexical calculation and meaning given by language common sense. 

Since there are serval kinds of constructions of degree structures mentioned in first section which are comparative meaning and assignable meaning, a lexical should have ability to give a interpretation of all these structures. \ref{john_gao_2_mi} gives a assignable meaning form example, and it's lexical entry calculation is shown in \ref{john_gao_2_mi_LE}, who uses traditional gradable adjectives lexical type in \ref{von_stechow_LE}. To resolve $\lambda$ reduction, the lexical entry of \textit{tall} needs a measure phrase with lexical entry $d$ and an individual $x$ , which are respectively \textit{2 meters} and \textit{John}. \textit{2 meters} changes the semantic type of \textit{tall} from $<d,<e,t>>$ to $<e,t>$, and \textit{John} is input as type $e$ which makes final result to type $t$. The result of lexical entry calculation tells a truth that John's height is not only equal to 2 meters precisely, but also has a possibility to greater than 2 meters. But actually we do know that sentence in \ref{john_gao_2_mi} means John's height is 2 meters accurately, which is mismatch with the result of semantic calculation. The reason why this mistake is made is that the partial ordering relation of degree is encoded in gradable adjectives, \colorbox{yellow}{and there is no way to resolve this ``greater equal''}. \ref{tallLE_b} will also lead this problem, since the lexical entry of the combination of null operator and gradable adjective is same with the lexical entry shown in \ref{von_stechow_LE}.

\begin{enumerate}
    \item \label{john_gao_2_mi} Jhon is 2 meters tall.
\end{enumerate}

\begin{enumerate}
    \item \label{john_gao_2_mi_LE}
    
    \begin{enumerate}
    \item \label{john_gao_2_mi_LE_a} 
    $[\![tall]\!] = \lambda d \lambda x.[Height(x) \geq d]$
    
    \item \label{john_gao_2_mi_LE_b} 
    $\begin{aligned}[t]
        [\![2 \enspace meters \enspace tall]\!] &= [\![tall]\!](2m) \\
        &= \lambda x.[Height(x) \geq 2m]
    \end{aligned}$
    
    \item \label{john_gao_2_mi_LE_c} 
    $\begin{aligned}[t]
        [\![John \enspace is \enspace 2 \enspace meters \enspace tall]\!] &= [\![tall \enspace 2m]\!](John) \\
        &= Height(John) \geq 2m
    \end{aligned}$
    
    \end{enumerate}
\end{enumerate}

To fix this problem, this paper propose that the lexical entries of gradable adjective in superiority meaning and assignable meaning should be different. The detail will be discussed in next section.


\section{MANDARIN DEGREE SEMANTIC ANALYSIS}

\noindent
In this section, this paper will do serval works to give a semantic and syntactic analyses of simple degree structure in Mandarin. At the first, based on former researchers' work, this paper will give lexical entry of gradable adjectives. After that, based on classification given in introduction, this paper will give a whole analyses respectively.

From this paper's view, different meaning of degree constructions should have different lexical entries of gradable adjectives. Because the function of gradable adjectives is totally different in superiority meaning and assignable meaning. As discussed in introduction, a degree structure is superiority meaning when gradable adjective bears comparison function, so this paper adopt the partial ordering definition of lexical entry, which is re-wrote in \ref{tallLE_re_a}; and on the opposite, a degree structure is assignable meaning when gradable adjective bears assignment function, thus the lexical entry of gradable adjective should have a equal sign, which is shown in \ref{tallLE_re_b}.

Another reason which supports the separation of lexical entries is the mistake mentioned in last section. If a positive form has ``greater than'' meaning in gradable adjective, a wrong result will be calculated which makes \textit{John is 2 meters tall} expresses \textit{John is more than 2 meters tall}.

\begin{enumerate}
    \item \label{tallLE_re}
    
    \begin{enumerate}
        \item \label{tallLE_re_a} 
        $[\![tall]\!]=\lambda d \lambda x.[Height(x) \geq d]$
    
        \item \label{tallLE_re_b} 
        $[\![tall]\!]=\lambda d \lambda x.[Height(x) = d]$
    
    \end{enumerate}
\end{enumerate}

Next, follow the classification given in the introduction, the syntactic structure and semantic calculation will be given respectively in follow parts.

\subsection{Comparative meaning}

\noindent
At the beginning, here we propose that there are two types syntactic structures in comparative meaning. For convenient, we give the two layers structure a name ``DegP-AP'' structure, and three layers structure a name ``DegP-DegP-AP''. 

In the ``DegP-DegP-AP'' structure, the head of the higher DegP functions to introduce the standard, and the head of the lower DegP functions to introduce the difference between the comparee and the standard which is denoted by a differential phrase. The gradable adjective is the head of the adjective phrase. For the semantic part, given the DegP hypothesis illustrated above, the higher DegP is obligatory in that, the extra argument $d$ should be bound by a functional projection which ultimately shifts the semantic type of gradable adjectives form $<d,<e,t>>$ to $<e,t>$, in order to absorb the comparee typed $e$ successfully. The differential phrase is not required by the lexical entry of gradable adjectives, which can also support its optional appearance. From our intuition, the non-appearance of a standard (explicit or implicit) leads to the failure of expressing the comparative meaning, in contrast, the difference between the comparee and the standard should in no way be expressed consistently. In addition, the occurrence of the difference between two objects naturally requires the the occurrence of the two objects to be compared as a prerequisite. Here we assume that the lower DegP is projected if and only if there is an explicit differential phrase and the differential phrase is strictly greater than zero. The explanations of this assumption will be elaborated immediately on the below. In addition, the non-appearance of a projection when it is not needed in completing the meaning of an utterance also satisfies the economical principal of language.

We further classifies the comparative meaning into three distinct subtypes, which are positivity, superiority and equality. We will unfold the syntactic structures and the semantic interpretations of these three subtypes below.

\subsubsection{Positivity}

\noindent
Firstly, a example of comparative of positivity is shown in \ref{positivity_example_1}. Literaturally, it is classified into the positive form. In this paper, we identify this sentence as positivity ushered in the comparative meaning, which is significantly from previous analyses. The reason this sentence is classified as expressing kind of comparative meaning is that the positive judgment of this sentence is based on a comparison between the height of \textit{John} and a value of height considered as the standard by people in mind, which needs not to be indicated or inferred from the context. In fact, this standard value is encoded into the comparative morpheme in positivity sentences. It deserves some notice that positivity is different from comparative of superiority with implicit standard, in that the implicit standard can be inferred from the context rather than encoded by the comparative morpheme itself. The contrast between the sentences in \ref{positivity_example_2_a} and \ref{positivity_example_2_b} can well reveal this point, where the implicit standard is represented as ``Pro'', which can be replaced with the corresponding expression from the context. The meaning of \ref{positivity_example_2_a} and \ref{positivity_example_2_b} is exactly the same, which can also demonstrate the correctness of identifying the implicit standard as ``Pro''. In contrast, we cannot reconstruct the explicit standard from the context for comparative of positivity which explains for the illegitimacy of the sentence in \ref{positivity_example_3_a}.

\begin{enumerate}
    \item \label{positivity_example_1}
    John hen gao.  \\
    John very tall \\
    `John is very tall'.
\end{enumerate}

\begin{enumerate}
    \item \label{positivity_example_2}
    \begin{enumerate}
        \item \label{positivity_example_2_a}
        John he Mary shui gao? \\
        John and Mary who tall? \\
        `Who is taller between John and Mary?' \\
        - John Pro gao.\\
        - John Pro tall. \\
        - `John is taller'.

        \item \label{positivity_example_2_b}
        John he Mary shui gao? \\
        John and Mary who tall? \\
        `Who is taller between John and Mary?' \\
        - John bi Mary gao.\\
        - John than Mary tall. \\
        - `John is taller than Mary'.

    \end{enumerate}
\end{enumerate}

\begin{enumerate}
    \item \label{positivity_example_3}
    \begin{enumerate}
        \item \label{positivity_example_3_a}
        * John hen Pro gao.  \\
        \hspace*{0.5em} John very Pro tall \\
        \hspace*{0.5em} `John is very Pro tall'.

        \item \label{positivity_example_3_b}
        John hen gao.  \\
        John very tall \\
        `John is very tall'.
    \end{enumerate}
\end{enumerate}

According to the assumption about one functional layer or two functional layers on the beginning of this section, given no appearance of DiffP in comparative of positivity, one layer of DegP above AP is considered as the correct syntactic structure for comparative of positivity, which is illustrated as below.

\begin{enumerate}
    \item $John [_{DegP} hen \enspace \xi [_{AP} gao]]$
\end{enumerate}

\begin{figure}[H]
    \centering
    \includegraphics[width=0.5\textwidth]{pic/positive_structure.png}
    \begin{caption}
        \\ \vspace{-1.1ex}
        Structure for Positivity.
    \end{caption}
\end{figure}

The null nature of DegP head should be licensed by the corresponding expression, which is undertaken by the obligatory \textit{hen}(`very') in this sentence, hence occupies the specifier position of DegP. The obligatoriness of \textit{hen}-like degree adverbs (\textit{feichang}(`much'), \textit{ting}(`very'), etc.) can be reflected by the ungrammatical of the sentence in \ref{positivity_example_5}.

\begin{enumerate}
    \item \label{positivity_example_5}
    * John gao.  \\
    \hspace*{0.5em} John tall \\
    \hspace*{0.5em} John is tall'.
\end{enumerate}

However, what should be pointed out is that \textit{hen}(`very') is not prohibited from dropping in other sentences involving positivity. As shown by sentences \ref{positivity_example_6}, \textit{hen}-like degree adverbs are allowed to drop.

\begin{enumerate}
    \item \label{positivity_example_6}
    \begin{enumerate}
        \item \label{positivity_example_6_a}
        John bu gao. \\
        John not tall. \\
        `John is not tall' 

        \item \label{positivity_example_6_b}
        John gao ma? \\
        John tall? \\
        `Is John tall?' 

        \item \label{positivity_example_6_c}
        John gao, Mary ai. \\
        John tall, Mary short. \\
        `John is tall, Mary is short.' 

    \end{enumerate}
\end{enumerate}

This paper will not go into the reasons why sentences like above allow the dropping of \textit{hen}(`very') deeply because they are not canonical sentences, which go beyond the scope of this research. In brief, the sentences above involve other elements related to degree, for example, negative adverb, sentence final particles, focus constructions, etc., which can also license the null degree head. \hl{Ref}

Now let us turn to the semantics of comparative of positivity. \hl{Ref} represents the lexical entry of the degree head in positivity as \ref{positivity_example_7_a}, where $d_{stnd}$ refers to the value of corresponding dimension considered as the standard by speakers in mind. With the standard value encoded, the degree head $\xi$ of comparative of positivity has both evaluative function and type-shift function \hl{Ref}. Given the equivalence relation which is proven in \ref{positivity_example_7_b}, this paper simplifies the lexical entry of the degree head as \ref{positivity_example_8_a}.

\begin{enumerate}
    \item \label{positivity_example_7}
    \begin{enumerate}
        \item \label{positivity_example_7_a}
        $[\![\xi]\!]=\lambda G \lambda x. \exists d [G(x,d) \land (d=d_{stnd})] \qquad <d,<e,t>,<e,t>>$

        \item \label{positivity_example_7_b}
        $\exists d [G(x,d) \land (d=d_{stnd})] \Leftrightarrow G(x,d_{stnd}) \enspace where \enspace G(x,d) = G-degree(x) \geq d$

    \end{enumerate}
\end{enumerate}

With the lexical entry of the gradable adjective \textit{tall} given in \ref{tallLE_re_a}, the meaning of the sentence in \ref{positivity_example_1} can be figured out successfully following the semantic calculation illustrated in \ref{positivity_example_8}, which is paraphrases as \ref{positivity_example_8_d}.

\begin{enumerate}
    \item \label{positivity_example_8}
    \begin{enumerate}
        \item \label{positivity_example_8_a}
        $[\![\xi]\!]=\lambda G \lambda x. [G(x,d_{stnd})] \qquad <d,<e,t>,<e,t>>$

        \item \label{positivity_example_8_b}
        $\begin{aligned}[t]
            [\![\xi \enspace gao]\!] &= \lambda G \lambda x. [G(x,d_{stnd})]([\![gao]\!]) \\
            &= \lambda x.[height(x) \geq d_{stnd}] \qquad <e,t>
        \end{aligned}$

        \item \label{positivity_example_8_c}
        $\begin{aligned}[t]
            [\![John \enspace \xi \enspace gao]\!] &= \lambda x.[height(x) \geq d_{stnd}]([\![John]\!]) \\
            &= height(John) \geq d_{stnd} \qquad <t>
        \end{aligned}$

        \item \label{positivity_example_8_d}
        \ref{positivity_example_8_c} is paraphrased as ``the height of John exceeds the standard height considered by people in mind''

    \end{enumerate}
\end{enumerate}


\subsubsection{Superiority}

\noindent
According to the assumption in the beginning of this section, Superiority sentences without DiffP are assumed to project only one functional layer above AP, taking ``DegP-AP'' as their syntactic structure.

\ref{superiority_example_1} gives serval examples which are all classified as comparative of superiority. And the sentence \ref{superiority_example_1_a} therein is classified as superlative form in the literature, which is different from the classification method taken in this paper.

\begin{enumerate}
    \item \label{superiority_example_1}
    \begin{enumerate}
        \item \label{superiority_example_1_a}
        John bi 1.7 mi gao. \\
        John than 1.7 meters gao. \\
        `John is taller than 1.7 mi' 

        \item \label{superiority_example_1_b}
        John bi Mary gao. \\
        John than Mary tall. \\
        `John is taller than Mary.' 

        \item \label{superiority_example_1_c}
        John zai yiban zhong zui gao. \\
        John in class.one in most tall. \\
        `John is tallest in class one' 

    \end{enumerate}
\end{enumerate}

Although types of the standard in these three sentences above are different, they are all classified as comparative of superiority sentences expressing comparative meaning in that they all implicate some kind of comparison made along the dimension of scale denoted by the gradable adjective between the comparee and another element serving as the standard in the comparison, and the difference between degree along the dimension denoted by the gradable adjective of the comparee and of the standard is not equal to zero. Types of element serving as the standard in the comparison can be variable: a degree denoted by direct degree expression in \ref{superiority_structure}, an individual denoted by NP(DP) in \ref{superiority_example_2_b}, a set of individuals denoted by \hl{ji he ming ci} in \ref{superiority_example_2_c}.

\begin{enumerate}
    \item \label{superiority_structure}
    $[_{IP} [CompareeP] [_{I^{\prime}} I^{0} [_{...} [[_{DegP} [StndP] [_{Deg^{\prime}} \xi_{n} [_{AP} [_{A^{\prime}} A^{0}]]]]]]]] \qquad n \in \{1, 2, 3\}$
\end{enumerate}

\begin{figure}[H]
    \centering
    \includegraphics[width=0.6\textwidth]{pic/superiority_structure.png}
    \begin{caption}
        \\ \vspace{-1.1ex}
        Structure for Superiority with standard phrase.
    \end{caption}
\end{figure}

\begin{enumerate}
    \item \label{superiority_example_2}
    \begin{enumerate}
        \item \label{superiority_example_2_a}
        for $\xi_n$, $n=1$ if and only if the standard is a degree;

        \item \label{superiority_example_2_b}
        for $\xi_n$, $n=2$ if and only if the standard is an individual;

        \item \label{superiority_example_2_c}
        for $\xi_n$, $n=3$ if and only if the standard is a set of individuals;

    \end{enumerate}
\end{enumerate}

For the syntactic part, no matter what type of the standard is taken, the general structure is almost the same. For achieving a unifying analysis, we use $\xi_n$ to represent for the degree head in higher DegP, in which the index $n$ is to distinguish variants of the degree head. The syntactic structure of comparative of superiority without DiffP is illustrated in \ref{superiority_structure}, with clear account of the index of $\xi_n$ given in \ref{superiority_example_2}.

Some more words perhaps should be said about instantiation of the degree head. When the standard is a degree or an individual, the degree morpheme could be a null morpheme $\emptyset$, could have phonetic realization as \textit{geng}(`more'), \textit{hai}(`still'), etc, as shown in \ref{superiority_example_3_a}. It should be pointed out the null functional morpheme  always only has the neutral meaning, while its co-existent counterpart with phonetic realizations should have some minor distinct meaning, otherwise it is redundant to use  different morphemes to undertake the single one function as well as express exactly the same meaning. Some \hl{Ref} argue that \textit{geng}(`more'), \textit{hai}(`still') are degree adverbs in Mandarin rather than degree morphemes. This paper do not agree with that perspective. Criteria for distinguishing functional morphemes and lexical morphemes are ...\hl{Ref} Besides, in the environment where the standard is implicit and there is no other potential licensors for the degree head, an overt degree head is obligatorily required, as shown in \ref{superiority_example_3_b}. Noticed that ``Pro'' cannot license a null functional morpheme \hl{Ref} and why sentence in \ref{positivity_example_2_a} of last sub section allows the null degree head is due to some other reason \hl{...?}. In fact, when the standard is implicit, it is \textit{geng}-like morphemes that indicates the comparative of superiority meaning of a sentence. However, when comes to the standard as a set of individuals, whether the standard is explicit or implicit, the degree morpheme should consistently realized as \textit{zui}(`most'), which hosts the neutral meaning, as shown by \ref{superiority_example_4}.

\begin{enumerate}
    \item \label{superiority_example_3}
    \begin{enumerate}
        \item \label{superiority_example_3_a}
        John bi Mary (geng/hai) gao. \\
        John than Mary much tall. \\
        `John is much taller than Mary'

        \item \label{superiority_example_3_b}
        John Pro *(geng) gao. \\
        John Pro *(much) tall.  \\
        `John is *(much) tall.'

    \end{enumerate}
\end{enumerate}

\begin{enumerate}
    \item \label{superiority_example_4}
    \begin{enumerate}
        \item \label{superiority_example_4_a}
        John za yiban zhong *(zui) gao. \\
        John in class.one *(most) tall. \\
        `John is tallest in class one'

        \item \label{superiority_example_4_b}
        John Pro *(zui) gao. \\
        John Pro *(most) tall.  \\
        `John is tallest.'

    \end{enumerate}
\end{enumerate}

When the degree head is a null morpheme $\emptyset$ or a bound morpheme with phonetic realization like \textit{geng}(more) and \textit{zui}(most), the gradable adjective moves up to combine with the degree head \hl{Ref}. The comparee denoted by NP(DP) is base generated in the specifier of AP, where no Case can be assigned. Therefore the comparee moves to the specifier of the IP to be assigned Nominative Case. \hl{do some lambda cal}

Then for the semantic part. According to Abney's DegP hypothesis, the functional projection DegP in comparative of superiority sentences consistently serves to bind the extra argument $d$ in the lexical entry of gradable adjectives by introducing the standard in comparative meaning. In terms of the function of the head of DegP, we make a significant  difference from previous scholars. The majority of previous analyses \hl{Ref} considers the comparative morpheme which occupies the head position of DegP bear the crucial ``greater than'' meaning in making comparison. However, the lexical entry of gradable adjectives shown in \ref{tallLE_re_a} which is repeated here as \ref{tallLE_re_re}, has already included ``the partial ordering relation'' which is established between the comparee and the standard along the dimension of scale denoted by the gradable adjective, which conveys the ``greater than'' meaning. Hence it is quite redundant to appoint another morpheme to denote the ``greater than'' meaning.  Here we assume that the functional head of DegP severs to usher in the standard as well as adjust the the method of absorbing the standard accordingly so that it be incorporated into the degree argument of the gradable. As a result, the semantic type and lexical entry of the head of DegP are variable corresponding to distinct standard types.

\begin{enumerate}
    \item \label{tallLE_re_re}
    $[\![tall]\!]=\lambda d \lambda x.[Height(x) \geq d] \qquad <d,<e,t>>$
\end{enumerate}

Semantics of these three variants $\xi_1$, $\xi_2$ and $\xi_3$ are elaborated as below.

Semantic calculations in \ref{superiority_example_5} corresponds to the sentence in \ref{superiority_example_1_a} where the standard is a degree typed $d$. The lexical entry of the degree head $\xi_1$ is given in \ref{superiority_example_5_a}. The output proposition is paraphrased as \ref{superiority_example_5_e}, which derives the meaning of the sentence correctly.

\begin{enumerate}
    \item \label{superiority_example_5}
    \begin{enumerate}
        \item \label{superiority_example_5_a}
        $[\![\xi_1]\!] = \lambda G \lambda d \lambda x.[G(x) \geq d] \qquad <<d,<e,t>>,<d,<e,t>>>$

        \item \label{superiority_example_5_b}
        $\begin{aligned}[t]
            [\![\xi_1 \enspace gao]\!] &= \lambda G \lambda d \lambda x.[G(x) \geq d]([\![gao]\!]) \\
            &= \lambda d \lambda x.[height(x) \geq d] \qquad <d,<e,t>>
        \end{aligned}$

        \item \label{superiority_example_5_c}
        $\begin{aligned}[t]
            [\![bi \enspace 1.7mi \enspace \xi_1 \enspace gao]\!] &= \lambda d \lambda x.[height(x) \geq d]([\![bi \enspace 1.7mi]\!]) \\
            &= \lambda d \lambda x.[height(x) \geq d](1.7mi) \\
            &= \lambda x.[height(x) \geq 1.7mi] \qquad <e,t>
        \end{aligned}$

        \item \label{superiority_example_5_d}
        $\begin{aligned}[t]
            [\![John \enspace bi \enspace 1.7mi \enspace \xi_1 \enspace gao]\!] &= \lambda x.[height(x) \geq 1.7mi](John) \\
            &= height(John) \geq 1.7mi \qquad <t>
        \end{aligned}$

        \item \label{superiority_example_5_e}
        \ref{superiority_example_5_d} is paraphrased as ``the height of John exceeds 1.7 metres''

    \end{enumerate}
\end{enumerate}

Semantic calculations in \ref{superiority_example_6} corresponds to the sentence in \ref{superiority_example_1_b} where the standard is an individual typed $e$. The lexical entry of the degree head $\xi_2$ is given in \ref{superiority_example_6_a}. The output proposition is paraphrased as \ref{superiority_example_6_e}, which derives the meaning of the sentence correctly.

\begin{enumerate}
    \item \label{superiority_example_6}
    \begin{enumerate}
        \item \label{superiority_example_6_a}
        $[\![\xi_2]\!] = \lambda G \lambda y \lambda x.[G(x,Max \enspace d_2(G(y,d_2)))] \qquad <<d,<e,t>>,<e,<e,t>>>$

        \item \label{superiority_example_6_b}
        $\begin{aligned}[t]
            [\![\xi_2 \enspace gao]\!] 
            &= \lambda G \lambda y \lambda x.[G(x,Max \enspace d_2(G(y,d_2)))]([\![gao]\!]) \\
            &= \lambda y \lambda x.[height(x) \geq Max \enspace d_2(height(y) \geq d_2)] \qquad <e,<e,t>>
        \end{aligned}$

        \item \label{superiority_example_6_c}
        $\begin{aligned}[t]
            [\![bi \enspace Mary \enspace & \xi_2 \enspace gao]\!] \\
            &= \lambda y \lambda x.[height(x) \geq Max \enspace d_2(height(y) \geq d_2)]([\![bi \enspace Mary]\!]) \\
            &= \lambda y \lambda x.[height(x) \geq Max \enspace d_2(height(y) \geq d_2)](Mary) \\
            &= \lambda x.[height(x) \geq Max \enspace d_2(height(Mary) \geq d_2)] \qquad <e,t>
        \end{aligned}$

        \item \label{superiority_example_6_d}
        $\begin{aligned}[t]
            [\![John \enspace bi \enspace & Mary \enspace \xi_2 \enspace gao]\!] \\
            &= \lambda x.[height(x) \geq Max \enspace d_2(height(Mary) \geq d_2)](John) \\
            &= height(John) \geq Max \enspace d_2(height(Mary) \geq d_2 ) \qquad <t>
        \end{aligned}$

        \item \label{superiority_example_6_e}
        \ref{superiority_example_6_d} is paraphrased as ``the height of John exceeds the height of Mary''

    \end{enumerate}
\end{enumerate}

Semantic calculations in \ref{superiority_example_7} corresponds to the sentence in \ref{superiority_example_1_c} where the standard is set of individual typed $<e,t>$. The lexical entry of the degree head $\xi_3$ is given in \ref{superiority_example_7_a}. The output proposition is paraphrased as \ref{superiority_example_7_e}, which derives the meaning of the sentence correctly.

\begin{enumerate}
    \item \label{superiority_example_7}
    \begin{enumerate}
        \item \label{superiority_example_7_a}
        $\begin{aligned}[t]
            [\![\xi_3]\!] 
            &= \lambda G \lambda C \lambda x.[G(x,Max \enspace d_2(G(C,d_2)))] \\
            & \qquad <<d,<e,t>>,<<e,t>,<e,t>>>
        \end{aligned}$

        \item \label{superiority_example_7_b}
        $\begin{aligned}[t]
            [\![\xi_3 \enspace gao]\!] 
            &= \lambda G \lambda C \lambda x.[G(x,Max \enspace d_2(G(C,d_2)))]([\![gao]\!]) \\
            &= \lambda C \lambda x.[height(x) \geq Max \enspace d_2(height(C) \geq d_2)] \qquad <<e,t>,<e,t>>
        \end{aligned}$

        \item \label{superiority_example_7_c}
        $\begin{aligned}[t]
            [\![zai \enspace yiban \enspace & zhong \enspace \xi_3 \enspace gao]\!] \\
            &= \lambda C \lambda x.[height(x) \geq Max \enspace d_2(height(C) \geq d_2)]([\![zai \enspace yiban \enspace zhong]\!]) \\
            &= \lambda C \lambda x.[height(x) \geq Max \enspace d_2(height(C) \geq d_2)](yiban) \\
            &= \lambda x.[height(x) \geq Max \enspace d_2(height(yiban) \geq d_2)] \qquad <e,t>
        \end{aligned}$

        \item \label{superiority_example_7_d}
        $\begin{aligned}[t]
            [\![John \enspace zai \enspace yiban & \enspace zhong \enspace \xi_3 \enspace gao]\!] \\
            &= \lambda x.[height(x) \geq Max \enspace d_2(height(yiban) \geq d_2)](John) \\
            &= height(John) \geq Max \enspace d_2(height(yiban) \geq d_2) \qquad <t>
        \end{aligned}$

        \item \label{superiority_example_7_e}
        \ref{superiority_example_7_d} is paraphrased as ``the height of John exceeds or equal to the  maximum height of all students in class one''

    \end{enumerate}
\end{enumerate}

Based on the assumption given at the beginning of this section, sentences classified into comparative of superiority with DiffP own two functional layers above AP, taking ``DegP-DegP-AP'' as their syntactic structure. Examples are shown in \ref{superiority_example_8}:

\begin{enumerate}
    \item \label{superiority_example_8}
    \begin{enumerate}
        \item \label{superiority_example_8_a}
        John bi 1.7mi gao 2limi. \\
        John than 1.7meters tall 2 centimeters. \\
        `John is 2 centimeters taller than 1.7 meters.'

        \item \label{superiority_example_8_b}
        John bi Mary gao 2limi. \\
        John than Mary tall 2 centimeters.  \\
        `John is 2 centimeters taller than Mary.'

    \end{enumerate}
\end{enumerate}

Comparative of superiority with overt DiffP expression need an additional DegP to host this DiffP, which indicates the difference between the value of the standard and the comparee along the dimension of scale denoted by the gradable adjective. Hence, the functional head of the additional lower DegP serves to introduce this DiffP. As mentioned above, distinct from the obligatory projection of the higher DegP, this lower DegP projects iff DiffP is overtly expressed and DiffP is not equal to zero. Ungrammatical of the sentence in \ref{superiority_example_9_a} can be explained by the DiffP which is equal to zero. In fact, the meaning of \ref{superiority_example_9_a} can be well expressed by another kind of construction with no DiffP named Equality, as shown in \ref{superiority_example_9_b}. The details of comparative of equality like \ref{superiority_example_9_b} will be discussed in the next subsection.

\begin{enumerate}
    \item \label{superiority_example_9}
    \begin{enumerate}
        \item \label{superiority_example_9_a}
        * John bi 1.7mi gao 0limi. \\
        \hspace*{0.5em} John than 1.7meters tall 0 centimeters. \\
        \hspace*{0.5em} `John is 0 centimeters taller than 1.7 meters.'

        \item \label{superiority_example_9_b}
        John he Mary yiyang gao. \\
        John with Mary same tall.  \\
        `John is as tall as Mary.'

    \end{enumerate}
\end{enumerate}

Syntactic structure is the first priority, \ref{superiority_structure_with_diffP} gives the syntactic structure of comparative of superiority with DiffP. no matter what type of the standard is taken, the general structure is almost the same. The only difference from the comparative of superiority without DiffP is that superiority with DiffP needs an additional projection DegP, of which the specifier position is occupied by DiffP and the head is represented as $\mu$.

\begin{enumerate}
    \item \label{superiority_structure_with_diffP}
    $\begin{aligned}[t]
        [_{IP} [CompareeP] [_{I^{\prime}} I^{0} [_{...} [[_{DegP} [StndP] [_{Deg^{\prime}} \xi_{n} [_{DegP} [DiffP] [_{Deg^{\prime}} \mu [_{AP} & [_{A^{\prime}} A^{0}]]]]]]]]]] \\
        & n \in \{1, 2, 3\}
    \end{aligned}$
\end{enumerate}

\begin{figure}[H]
    \centering
    \includegraphics[width=0.7\textwidth]{pic/superiority_structure_diff.png}
    \begin{caption}
        \\ \vspace{-1.1ex}
        Structure for Superiority with differential phrase.
    \end{caption}
\end{figure}

\begin{enumerate}
    \item \label{superiority_example_10}
    \begin{enumerate}
        \item \label{superiority_example_10_a}
        for $\xi_n$, $n=1$ if and only if the standard is a degree;

        \item \label{superiority_example_10_b}
        for $\xi_n$, $n=2$ if and only if the standard is an individual;

        \item \label{superiority_example_10_c}
        for $\xi_n$, $n=3$ if and only if the standard is a set of individuals;

    \end{enumerate}
\end{enumerate}

Continuous head movements are involved. The head of AP first moves upwards to combine with the head of lower DegP, which is instantiated as a null morpheme $\emptyset$ or a bound morpheme with phonetic realization such as \textit{chu}(`exceed'), to form a compound, and then moves together to further combine with the head of the higher DegP when it is instantiated as a bound morpheme. Given that the specifier of AP is not a position to which Case can be assigned, the comparee denoted by NP(CP) moves to the specifier of TP in order to obtain its Nominative Case.

There is an alternative solution which inverse the hierarchy of the lower DegP and AP, which can be briefly represented as ``DegP-AP-DegP'', supported by many scholars \hl{ref}. This structure can also derive out the correct linear order of the sentence. which is displayed in \ref{superiority_example_11} (The structure shown in \ref{superiority_example_11} is slightly different from the original version. Here we do not care about unrelated technical details but just focus on the general structure of the hierarchy of the three layers, i.e., two DegPs and one AP), where AP take the lower DegP as its complement.

\begin{enumerate}
    \item \label{superiority_example_11}
    $\begin{aligned}[t]
        [_{IP} [CompareeP] [_{I^{\prime}} I^{0} [_{...} [[_{DegP} [StndP] [_{Deg^{\prime}} \xi [_{AP} [_{A^{\prime}} A^{0}]]]]]]]] \qquad n \in \{1, 2, 3\}
    \end{aligned}$
\end{enumerate}

In spite of the correct linear order derived, \hl{ref} has argued that the direction of head movements in the syntactic structure shown in \ref{superiority_example_11} is actually impossible. He claims that a functional head is an incorporated element rather than an incorporating element, giving rise to the impossibility of the movement from a functional degree head to a lexical adjective head. \hl{this ref} eliminate the alternative ``DegP-AP-DegP'' solution from a view of lexicon-syntax, while I propose another piece of evidence for eliminating the ``DegP-AP-DeP'' analysis from the perspective of formal semantics, which will be illustrated immediately below.

Then turn to semantic part of comparative of superiority with DiffP. The semantic interpretation of the higher DegP remains the same as that of the comparative of superiority without DiffP, so here we will not illustrate it in detail again. For the lower DegP, in this paper we assume that it undertake the mere function of measuring the difference between the value of the comparee and the value of  the standard along the dimension of scale denoted by the gradable adjective. It should be pointed out that this difference is expressed in the form of the absolute value for the reason that this difference itself indicates no ordering relation between the two values input into the comparative construction.  The ordering relation is encoded in the lexical entry of gradable adjectives,  indicated neither by the head of higher DegP (which serves merely to usher in the standard) nor by the head of lower DegP (which serves merely to introduce the absolute value of the difference between the two values compared). 

Most of  previous studied are inclined to appoint too heavy semantic function to one of the degree head and as a result make the semantic function of the other degree head marginal and transparent. What is worse, the ``greater than'' meaning encoded in the lexical entry of gradable adjective is seized by a degree head which should not have been in charge of indicating the ordering relation between the comparee and the standard. In addition, the transfer of the ``greater than'' meaning from gradable adjectives to the degree head would give rise to  wrong reading of sentences made out of gradable adjectives with DiffP. Here we illustrate this fatal disadvantage with an instance from von Stechow \hl{ref}.

This clear division of labor in terms of the semantic meaning bore by each component is crucial for the whole analysis and this perspective is also significantly different from previous analyses.

\subsubsection{Equality}

\begin{enumerate}
    \item
    \begin{enumerate}
        \item John he Mary yi yang gao.
    \end{enumerate}
\end{enumerate}

\subsection{Assignable meaning}

\begin{enumerate}
    \item
    \begin{enumerate}
        \item John gao 2 mi.
        \item John you 2 mi gao.
    \end{enumerate}
\end{enumerate}

\section{CONCLUSION}

\newpage

\printbibliography

\end{document}
